\documentclass{amsart}
\usepackage{graphicx}
\usepackage{epstopdf}
\usepackage[utf8]{inputenc}
\usepackage[swedish]{babel}
\usepackage[style=authoryear-ibid,backend=bibtex]{biblatex}
\usepackage{csquotes}
\usepackage{hyperref}
\addbibresource{bibliography.bib}
\title{Trängselskatt i Stockholm och Göteborg}
\author{Carl Brishammar \\ Olle Kjellvist \\ David Montgomery \\ Max Söderman}
\date{}

\begin{document}
\begin{abstract}
	Sed ut perspiciatis unde omnis iste natus error sit voluptatem accusantium doloremque laudantium, totam rem aperiam, eaque ipsa quae ab illo inventore veritatis et quasi architecto beatae vitae dicta sunt explicabo. Nemo enim ipsam voluptatem quia voluptas sit aspernatur aut odit aut fugit, sed quia consequuntur magni dolores eos qui ratione voluptatem sequi nesciunt. Neque porro quisquam est, qui dolorem ipsum quia dolor sit amet, consectetur, adipisci velit, sed quia non numquam eius modi tempora incidunt ut labore et dolore magnam aliquam quaerat voluptatem. Ut enim ad minima veniam, quis nostrum exercitationem ullam corporis suscipit laboriosam, nisi ut aliquid ex ea commodi consequatur? Quis autem vel eum iure reprehenderit qui in ea voluptate velit esse quam nihil molestiae consequatur, vel illum qui dolorem eum fugiat quo voluptas nulla pariatur?
\end{abstract}
\keywords{Trängsel, infrastrukturfinansiering, pengar, folkomröstning}
\maketitle
\newpage
\tableofcontents
\newpage
\section{Introduktion}
This is the way referencing works. \parencite{book}
You can also reference an article. As never seen in \textcite{article}...
\input{Introduktion.tex}
\section{Resultat}
\subsection{Hälsoeffekter och rekommendationer}
\subsubsection{Partiklar}
Partiklar med en diameter mindre än $10 \mu g$ (pm$_{10}$) kan komma ner och stanna i lungorna. Att utsättas för pm$_{10}$ innebär en ökad risk för att utväckla hjärt/kärlsjukdomar, andingssjukdomar samt lungcancer. Det finns ett tydligt samband mellan exponering av både pm$_{10}$ och pm$_{2,5}$ (partiklar med diameter mindre än $2,5 \mu g$) och förtida död. Det gäller också att en minskad exponering sänker dödligheten. WHO har därför satt sina rekommendarade gränsvärden för årsmedelvärde till $10 \mu g/m^3$ för pm$_{2,5}$ och till $20 \mu g/m^3$ för pm$_{10}$.
\cite{whoAir}
\begin{figure}
	\centering
	\includegraphics[width=.8\textwidth]{Bilder/pm25sth}
	\caption{Data från \url{http://slb.nu/slbanalys/historiska-data-luft/}}
	\label{fig:pm25sth}
\end{figure}

\begin{figure}
	\centering
	\includegraphics[width=.8\textwidth]{Bilder/pm10sth}
	\caption{Data från \url{http://slb.nu/slbanalys/historiska-data-luft/}}
	\label{fig:pm10sth}
\end{figure}

\begin{figure}
	\centering
	\includegraphics[width=.8\textwidth]{Bilder/pm10gbg}
	\caption{\cite{gbg}}
	\label{fig:pm10gbg}
\end{figure}
\subsection{Kvävedioxid}
Enligt \cite{whoAir} så finns det samband mellan en ökning av bronkit hos barn med astma och långvarig exponering för kvävedioxid. Samband finns också mellan minskad lungutveckling och NO$_2$ i koncentrationer som återfinns i europeiska städer idag.
\begin{figure}
	\centering
	\includegraphics[width=.8\textwidth]{Bilder/NO2sth}
	\caption{Data från \url{http://slb.nu/slbanalys/historiska-data-luft/}}
	\label{fig:NO2sth}
\end{figure}

\begin{figure}
	\centering
	\includegraphics[width=.8\textwidth]{Bilder/NO2gbg}
	\caption{\cite{gbg}}
	\label{fig:NO2gbg}
\end{figure}
\subsection{Marknära ozon}
WHO sänkte sina riktlinjer från $120 \mu g/m^3$ till $100 \mu g/m^3$, för ett 8h glidande medelvärde, 2005. Detta för att höga halter av ozon kan ge andningsbesvär, astma, och leda till lungsjukdomar. Enligt \cite{whoAir} så leder en ökning i exponering med $10 \mu g/m^3$ till att dödligheten ökar med $0,3\%$.
\begin{figure}
	\centering
	\includegraphics[width=.8\textwidth]{Bilder/ozone}
	\caption{Data från \url{http://slb.nu/slbanalys/historiska-data-luft/}}
	\label{fig:Ozone}
\end{figure}

\section{Diskussion}
\input{Diskussion.tex}
\section{Slutsats}
\input{Slutsats.tex}
\newpage
\printbibliography
\end{document}
