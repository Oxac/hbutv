\documentclass{amsart}

\makeatletter
\newenvironment{chapquote}[2][2em]
  {\setlength{\@tempdima}{#1}%
   \def\chapquote@author{#2}%
   \parshape 1 \@tempdima \dimexpr\textwidth-2\@tempdima\relax%
   \itshape}
  {\par\normalfont\hfill--\ \chapquote@author\hspace*{\@tempdima}\par\bigskip}
\makeatother


\usepackage[utf8]{inputenc}
\usepackage[swedish]{babel}
\usepackage[style=authoryear-ibid,backend=bibtex]{biblatex}
\usepackage{csquotes}
\addbibresource{bibliography.bib}
\title{Trängselskatt i Stockholm och Göteborg}
\author{Carl Brishammar \\ Olle Kjellvist \\ David Montgomery \\ Max Söderman}
\date{}

\begin{document}
\maketitle
\begin{abstract}
	Sed ut perspiciatis unde omnis iste natus error sit voluptatem accusantium doloremque laudantium, totam rem aperiam, eaque ipsa quae ab illo inventore veritatis et quasi architecto beatae vitae dicta sunt explicabo. Nemo enim ipsam voluptatem quia voluptas sit aspernatur aut odit aut fugit, sed quia consequuntur magni dolores eos qui ratione voluptatem sequi nesciunt. Neque porro quisquam est, qui dolorem ipsum quia dolor sit amet, consectetur, adipisci velit, sed quia non numquam eius modi tempora incidunt ut labore et dolore magnam aliquam quaerat voluptatem. Ut enim ad minima veniam, quis nostrum exercitationem ullam corporis suscipit laboriosam, nisi ut aliquid ex ea commodi consequatur? Quis autem vel eum iure reprehenderit qui in ea voluptate velit esse quam nihil molestiae consequatur, vel illum qui dolorem eum fugiat quo voluptas nulla pariatur?
\end{abstract}
\keywords{Trängsel, infrastrukturfinansiering, pengar, folkomröstning}
\newpage
\tableofcontents
\newpage
\section{Introduktion}

Hallå?

\begin{chapquote}{Från Transportstyrelsens hemsida}
``Trängselskatten syftar till att minska trängseln, förbättra miljön och bidra till att finansiera infrastruktursatsningar.”'
\end{chapquote}




%This is the way referencing works. \parencite{book}
%You can also reference an article. As never seen in \textcite{article}...
\begin{chapquote}{Lewis Carroll, \textit{Alice in Wonderland}}
``Begin at the beginning,'' the King said, gravely, ``and go on till you
come to an end; then stop.''
\end{chapquote}

\section{Diskussion}
\input{Diskussion.tex}
\section{Slutsats}
\input{Slutsats.tex}
\newpage
\printbibliography
\end{document}
